\section{Question 1}

I used the following code to make the network in the Figure \ref{fig:fig1}.

\begin{verbatim}
    library(igraph)
    g1 <- graph.formula(1-3,1-4,1-5,1-6,2-4,2-5,2-6,3-5,4-6)
    igraph.options(vertex.color="dodgerblue",vertex.size=25,
                   vertex.label.cex=1.25,vertex.label.color="white",
                   edge.color="red",edge.arrow.size=1,edge.width=2)
    tkplot(g1)
\end{verbatim}

\begin{figure}[H]
        \centering
        \includegraphics[width=0.5\textwidth]{Figure1.eps}
        \caption{\label{fig:fig1}Network for Question 1}
\end{figure}

\subsection{Part a}

\begin{answer}
    There are $6$ vertices and $9$ edges in the network.
    
    I checked my result in R:
    \begin{verbatim}
        # Find the number of edges and the number of vertices
        vcount(g1)
        ecount(g1)
        
        Result:
        [1] 6
        [1] 9
    \end{verbatim}
\end{answer}

\subsection{Part b}

\begin{answer}
    \begin{table}[H]
    \centering
    \caption{degree and degree distribution of nodes}
    \label{tab:tab1}
        \begin{tabular}{|c|c|}
            \hline
            \textbf{nodes} & \textbf{degree} \\ \hline
            1              & 4               \\ \hline
            2              & 3               \\ \hline
            3              & 2               \\ \hline
            4              & 3               \\ \hline
            5              & 3               \\ \hline
            6              & 3               \\ \hline
        \end{tabular}
        \quad
        \begin{tabular}{|c|c|}
        \hline
        \textbf{degree} & \textbf{frequency} \\ \hline
        0               & 0                  \\ \hline
        1               & 0                  \\ \hline
        2               & 0.16667            \\ \hline
        3               & 0.66667            \\ \hline
        4               & 0.16667            \\ \hline
        \end{tabular}
    \end{table}
    I used the code below in R to check my result
    \begin{verbatim}
        # Find the degrees
        degree(g1)
        # Find the degree distribution for the network g1
        degree.distribution(g1)
        
        Result:
        1 3 4 5 6 2 
        4 2 3 3 3 3 
        [1] 0.0000000 0.0000000 0.1666667 0.6666667 0.1666667
    \end{verbatim}
\end{answer}

\subsection{Part c}

\begin{answer}
    The neighbours of node $2$ are $4$, $5$, and $6$, and they have $3$, $3$, and $3$ neighbours separately. Hence, node $2$ has nearest neighbour degree of $\tfrac{3+3+3}{3} = 3$. 
    
    The neighbours of node $3$ are $1$, and $5$, and they have $4$, and $3$ neighbours separately. Hence, node $3$ has nearest neighbour degree of $\tfrac{4+3}{2} = 3.5$.
    
    The neighbours of node $4$ are $1$, $2$, and $6$, and they have $4$, $3$, and $3$ neighbours separately. Hence, node $4$ has nearest neighbour degree of $\tfrac{4+3+3}{3} \approx 3.333$.
    
    I checked my result in R using the function \verb+knn+
    \begin{verbatim}
        # Find the average nearest neighbour degree for vertices 2, 3, and 4
        knn(g1)$knn
        
        Result:
               1        3        4        5        6        2 
            2.750000 3.500000 3.333333 3.000000 3.333333 3.000000 
    \end{verbatim}
\end{answer}

\subsection{Part d}\label{sec:sec1d}

\begin{answer}
    Since the largest shortest path distance in the network is $2$. 
    
    R code is the following
    \begin{verbatim}
        # Find the diameter of the network
        diameter(g1)
        
        Result:
        [1] 2
    \end{verbatim}
    The pair $(5,6)$ has largest shortest path distance between them, and the pair $(1,2)$ has largest shortest path distance between them.
    I used the code below in R to return the pairs of nodes has largest shortest path distance between them 
    \begin{verbatim}
        # Find which pairs of nodes has largest shortest path distance 
        between them
        shortest.paths(g1)
        
        Result:
          1 3 4 5 6 2
        1 0 1 1 1 1 2
        3 1 0 2 1 2 2
        4 1 2 0 2 1 1
        5 1 1 2 0 2 1
        6 1 2 1 2 0 1
        2 2 2 1 1 1 0
    \end{verbatim}
    Clearly, the pairs $(1,2)$ and $(5,6)$ has distance $2$.
\end{answer}

\subsection{Part e}

\begin{answer}
    I would remove the edge between $1$ and $4$ to increase the diameter by one, since the largest shortest path distance between $3$ and $4$ would be $3$.
    
    I used the following code to check my result in R
    \begin{verbatim}
        newg1 <- graph.formula(1-3,1-5,1-6,2-4,2-5,2-6,3-5,4-6)
        igraph.options(vertex.color="dodgerblue",vertex.size=25,
                       vertex.label.cex=1.25,vertex.label.color="white",
                       edge.color="red",edge.arrow.size=1,edge.width=2)
        diameter(newg1)
        
        Result:
        [1] 3
    \end{verbatim}
\end{answer}

\subsection{Part f}

\begin{answer}
    Using the result from Part d, we can find the sum of shortest distance of the paths from each node in the network
    \begin{table}[H]
    \centering
    \caption{Table of closeness centralities for nodes in the network}
    \label{tab:tab2}
    \begin{tabular}{|c|c|c|}
    \hline
    \textbf{node} & \textbf{sum of shortest distances} & \textbf{closeness centrality} \\ \hline
    1             & 6                                  & 1/6                             \\ \hline
    2             & 8                                  & 1/8                             \\ \hline
    3             & 7                                  & 1/7                             \\ \hline
    4             & 7                                  & 1/7                             \\ \hline
    5             & 7                                  & 1/7                             \\ \hline
    6             & 7                                  & 1/7                             \\ \hline
    \end{tabular}
    \end{table}
    Hence the node $1$ has closeness centrality of $\tfrac{1}{6}$, and the node $2$ has closeness centrality of $\tfrac{1}{8}$.
    
    The R code I used to check my answer is below:
    \begin{verbatim}
        # Find the closeness centralities of the nodes
        closeness(g1)
        
        Result:
                1         3         4         5         6         2 
            0.1666667 0.1250000 0.1428571 0.1428571 0.1428571 0.1428571 
    \end{verbatim}
\end{answer}